% !TEX TS-program = xelatex
% !TEX encoding = UTF-8 Unicode
% !Mode:: "TeX:UTF-8"

\documentclass{resume}
\usepackage{zh_CN-Adobefonts_external} % Simplified Chinese Support using external fonts (./fonts/zh_CN-Adobe/)
%\usepackage{zh_CN-Adobefonts_internal} % Simplified Chinese Support using system fonts
\usepackage{linespacing_fix} % disable extra space before next section
\usepackage{cite}

\begin{document}
\pagenumbering{gobble} % suppress displaying page number

\name{戴永晋}

\basicInfo{
  \email{yongjindai09@gmail.com} \textperiodcentered\
  \phone{(+86) 158-5072-7131} \textperiodcentered\
  \wechat{goldenginkgo} \textperiodcentered\
  \linkedin[yongjindai]{https://www.linkedin.com/in/yongjindai}}

\section{\faCogs\ IT 技能}
% increase linespacing [parsep=0.5ex]
\begin{itemize}[parsep=0.5ex]
%  \item 编程语言: C == Python > C++ > Java
  \item 后端: Ruby, Ruby on Rails, Python, Java.
  \item 前端: HTML, CSS, JavaScript, jQuery, JSON, Bootstrap, Sass.
  \item 数据库: SQL/NoSQL, PostgreSQL, Redis.
  \item 测试: TDD, BDD, Jenkins, RSpec, Capybara, Cucumber, Selenium.
  \item 版本管理: Git, SVN.
  \item 系统管理: Linux (Red Hat, CentOS, Ubuntu), Windows Server.
  \item 虚拟化和云平台: VMware vSphere, KVM, OpenStack.
\end{itemize}

\section{\faUsers\ 工作经历}
\datedsubsection{\textbf{南京富士通南大软件技术有限公司} 南京}{2011 -- 2016}
\role{软件开发工程师}{}
\begin{itemize}
  \item 实施软件需求分析, 功能设计, 编码, 环境搭建与配置,以及测试.
  \item 维护遗留系统并解决客户问题.
\end{itemize}

\section{\faTasks\ 项目经验}
\datedsubsection{\textbf{Fujitsu Cloud Service K5 (K5)}}{2015 -- 2016}
\role{Ruby on Rails, Python}{}
\begin{onehalfspacing}
K5基于OpenStack和CloudFoundry, 是富士通的公有云平台, 类似AWS.
\begin{itemize}
  \item 为Region Manager (管理OpenStack的Web应用) 开发RESTful API接口.
  \item 安装配置高可用的OpenStack环境, 调查并提出了不停止虚拟机实例的条件下升级K5中OpenStack版本的方案.
  \item 解决了超过30个OpenStack的配置和使用问题, 包括4个关于虚拟机实例热迁移和OpenStack Heat的源码bug, 并为OpenStack社区贡献了3个文档bug的patch.
\end{itemize}
\end{onehalfspacing}

\datedsubsection{\textbf{Fujitsu ServerView Resource Orchestrator Cloud Edition (ROR CE)}}{2011 -- 2015}
\role{Ruby on Rails, Python}{}
\begin{onehalfspacing}
ROR CE是富士通的私有云解决方案, 通过有效利用服务器、存储、网络等ICT资源并提高运行和管理效率, 来实现ICT成本的最优化.
\begin{itemize}
  \item 设计并实现了新的灾难恢复方案来保护IT服务的可持续性, 该方案被超过5个大客户采用.
  \item 开发了VMware vCenter模拟器, 并利用该模拟器实施大规模测试.
  \item 性能调优, 重构代码, 使恢复时间目标(recovery time objective, RTO)从数天减少到数小时.
\end{itemize}
\end{onehalfspacing}

% Reference Test
%\datedsubsection{\textbf{Paper Title\cite{zaharia2012resilient}}}{May. 2015}
%An xxx optimized for xxx\cite{verma2015large}
%\begin{itemize}
%  \item main contribution
%\end{itemize}

\section{\faGraduationCap\  教育背景}
\datedsubsection{\textbf{南京大学}, 南京, 江苏}{2007 -- 2011}
\textit{学士}\ 软件工程
\begin{itemize}
  \item 人民奖学金三等奖
  \item 国家励志奖学金
\end{itemize}

%\section{\faHeartO\ 获奖情况}
%\datedline{\textit{第一名}, xxx 比赛}{2013 年6 月}
%\datedline{其他奖项}{2015}

\section{\faInfo\ 其他}
% increase linespacing [parsep=0.5ex]
\begin{itemize}[parsep=0.5ex]
%  \item 技术博客: http://goldginkgo.me/
%  \item GitHub: https://github.com/goldginkgo
  \item 语言: 英语 - 熟练(大学英语六级, 雅思 6.5)
\end{itemize}

%% Reference
%\newpage
%\bibliographystyle{IEEETran}
%\bibliography{mycite}
\end{document}
